\section{Matice monodromie a periodická řešení}\label{sec:monodromy_matrix_and_periodic_solutions}

\begin{thrm}\label{thrm:monodromy_matrix}
	Nechť matice $A(t)$ je $T$-periodická. Pokud $\Phi(t)$ je fundamentální, pak $\Phi(t+T)$ je také fundamentální a navíc existuje regulární konstantní matice $B$ taková, že platí následující dva body:
	
	\begin{enumerate}
		\item $\Phi(t+T) = \Phi(t) \cdot B$;
		\item $\mathrm{det} (B) = \exp \big( \int_{0}^{T} \mathrm{tr} (A(s)) ds \big) $.
	\end{enumerate}

	\begin{proof}
		Nejdříve ukážeme, že $\Phi(t+T)$ je fundamentální: označíme-li $\Psi(t) \defeq \Phi(t+t)$, pak
		
		\begin{equation}
			\dot{\Psi}(t) = \dot{\Phi}(t+T) = A(t+T) \Phi(t+T) = A(t) \Psi(t),
		\end{equation}
		
		\noindent
		což znamená, že $\Psi(t) = \Phi(t+T)$ je skutečně fundamentální.
		
		\medskip
		
		Dále bychom chtěli ukázat existenci matice $B$ a zbývající dva body tvrzení:
		
		\begin{enumerate}
			\item Ze základů lineární algebry víme, že určitě existuje regulární matice $B(t)$, která splňuje bod $1$ pro různé hodnoty $t$. Chceme ukázat, že $B(t) = B(t_{0}) \defeq B_{0} = \mathrm{const}$ pro libovolné fixní $t_{0}$. Lemma \ref{lm:1} implikuje, že $\Psi_{0}(t) \defeq \Phi(t) B_{0}$ je fundamentální. Zároveň ale $\Psi(t_{0}) = \Phi(t_{0}) B(t_{0}) = \Phi(t_{0}) B_{0} = \Psi_{0}(t_{0})$ pro všechna $t_{0}$. Pak z věty o existenci a jednoznačnosti řešení lineární ODR plyne, že pro všechna $t$ platí:
			
			\begin{equation}
				\Phi(t) B_{0} = \Psi_{0} (t) = \Psi (t) = \Phi (t) B(t) \implies B(t) = B_{0} = \mathrm{const}. 
			\end{equation}
			
			\item Z prvního bodu jednoduše vyplývá, že $B = \Phi^{-1}(t) \Phi(t+T)$. Navíc pomocí lemmatu \ref{lm:2} dostaneme:
			
			\begin{equation}
				W(t+T) = W(t_{0}) \exp \Big( \int_{t_{0}}^{t} \mathrm{tr} (A(s)) ds +  \int_{t}^{t+T} \mathrm{tr} (A(s)) ds \Big) = W(t) \exp \Big( \int_{t}^{t+T} \mathrm{tr} (A(s)) ds \Big)
			\end{equation}
			
			Potom tedy platí následující: 
			
			\begin{equation}\label{eq:thrm1_prefinal}
				\mathrm{det} (B) = \frac{1}{ \mathrm{det} (\Phi(t)) } \cdot \mathrm{det} (\Phi(t+T)) = \frac{1}{W(t)} \cdot W(t+T) = \exp \Big( \int_{t}^{t+T} \mathrm{tr} (A(s)) ds \Big).
			\end{equation}
			
			Uděláme-li v \eqref{eq:thrm1_prefinal} substituci $ z = s - t $, dostaneme tvrzení bodu $2$.
		\end{enumerate}
	\end{proof}
\end{thrm}

\medskip

\begin{remark}\label{remark:monodromy_matrix_calculation}
	Pro jednoduchost výpočtu matice $B$ lze volit $t = 0$. Potom $B = \Phi^{-1}(0) \Phi(T)$, přičemž lineární nezávislost sloupců fundamentální matice $\Phi^{-1}(0)$ umožňuje zvolit takovou bázi prostoru, ve které $\Phi^{-1}(0) = I$. Potom $B = \Phi(T)$.
\end{remark}

\medskip

\begin{defn}\label{defn:monodromy_matrix} 
	(Matice monodromie) \\
	Matice $B$ z věty \ref{thrm:monodromy_matrix} se nazývá \textit{matice monodromie}, její vlastní čísla $\rho \in \sigma(B)$ se nazývají \textit{charakteristické multiplikátory}. Číslo $\lambda \in \mathbb{C}$ takové, že $\rho = e^{ \lambda t }$, $\rho \in \sigma(C)$, se nazývá \textit{charakteristický exponent}.
\end{defn}

\medskip

\begin{remark}\label{remark:monodromy_matrix_properties}
	(Vlastnosti matice monodromie)
	
	\begin{enumerate}
		\item Za podmínky $X(0) = I $ podle poznámky \ref{remark:monodromy_matrix_calculation} dostáváme 
		
		\begin{itemize}
			\item $\mathrm{det}(B) = \Pi_{i = 1}^{n} \rho_{i} = \exp \Big( \int_{0}^{T} \mathrm{tr} (A(s)) ds \Big)$;
			
			\item $\mathrm{tr}(B) = \sum_{i = 1}^{n} \rho_{i}$.
		\end{itemize}
		
		\item Charakteristické exponenty nejsou dány jednoznačně: pokud $\lambda$ je charakteristický exponent, pak $\lambda + i \tfrac{2 \pi m}{T}$, $m \in \mathbb{Z}$, je také charakteristický exponent.
		
		\item Charakteristické multiplikátory nezávisí na výběru fundamentální matice.
		
		\begin{proof}
			Nechť $\Phi(t)$ a $\hat{\Phi}(t)$ jsou dvě různé fundamentální matice. Potom podle věty \ref{thrm:monodromy_matrix} platí, že existují regulární konstantní matice $B$ a $\hat{B}$ takové, že $\Phi(t+T) = \Phi(t) B$ a $\hat{\Phi}(t+T) = \hat{\Phi}(t) \hat{B}$.  Z důkazu té samé věty navíc vyplývá, že existuje taková regulární konstantní matice $C$, že $\hat{\Phi}(t) = \Phi(t) C$, pak
			
			\begin{equation*}
				\hat{\Phi}(t+T) = \Phi(t+T) C = \Phi(t) B C 
				\wedge
				\hat{\Phi}(t+T) = \hat{\Phi}(t) \hat{B} = \Phi(t) C \hat{B},
			\end{equation*}
			
			\noindent
			a tedy $\hat{B}  = C^{-1} B C$, tj. matice $\hat{B}$ a $B$ jsou podobné, a tudíž $\sigma(\hat{B}) = \sigma(B)$.
		\end{proof}
	\end{enumerate}
\end{remark}

\medskip

%\begin{exmp}
%	$$ 
%	A = 
%	\begin{pmatrix} 
%		0 & \omega \\ 
%		-\omega & 0
%	\end{pmatrix}
%	\implies
%	e^{tA} = 
%	\begin{pmatrix} 
%	\cos \omega t & \sin \omega t \\ 
%	-\sin \omega t & \cos \omega t
%	\end{pmatrix}. 
%	$$
%\end{exmp}

\begin{thrm}\label{thrm:solution_properties}
	Nechť $\rho$ je charakteristický multiplikátor a $\lambda$ je odpovídající mu charakteristický exponent. Potom existuje řešení $x(t)$ splňující následující:
	
	\begin{enumerate}
		\item $x(t+T) = \rho x(t)$;
		\item existuje $T$-periodická funkce $p(t)$ taková, že $x(t) = e^{ \lambda t } p(t)$.
	\end{enumerate} 

	\begin{proof}
		Díky větě \ref{thrm:monodromy_matrix} a definici \ref{defn:monodromy_matrix} víme, že $\rho$ je vlastní číslo konstantní regulární matice $B$ - označíme pomocí $b \in \mathbb{R}^{n}$ odpovídající tomuto vztahu vlastní vektor, pak $x(t) \defeq \Phi(t)b$. Potom platí
		
		\begin{equation}
			x(t+T) = \Phi(t+T) b = \{ \text{věta \ref{thrm:monodromy_matrix}} \} = \Phi(t) B b = \Phi(t) \rho b = \rho x(t), 
		\end{equation}
		
		\noindent
		což je tvrzení prvního bodu dané věty. Pro to, abychom dostali druhý bod, označíme $p(t) \defeq x(t) e^{ - \lambda t }$. Potřebujeme ukázat, že $p(t)$ má periodu $T$:
		
		\begin{equation}
			p(t+T) = x(t+T) e^{ - \lambda (t+T) } = \rho x(t) e^{ - \lambda t } \underbrace{e^{- \lambda T}}_{ = \tfrac{1}{\rho} } = x(t) e^{ - \lambda t } = p(t),
		\end{equation}
		
		\noindent
		tj. $p(t)$ je $T$-periodická.
	\end{proof}
\end{thrm}

\begin{remark}
	(Důsledky věty \ref{thrm:solution_properties})
	\begin{enumerate}
		\item Nechť $N \in \mathbb{N}$, pak $x(t+NT) = \rho^{N} x(t)$:
		
		\begin{itemize}
			\item pokud $\abs{\rho} < 1$, tj. $\mathrm{Re} (\lambda )< 0$, pak $x(t) \xrightarrow{t \rightarrow \infty} 0$;
			
			\item pokud $\abs{\rho} = 1$, tj. $\mathrm{Re} (\lambda) = 0$, pak $x(t)$ je \textit{pseudoperiodické} řešení (je periodické $\iff$ $\rho = \pm 1$ );
			
			\item pokud $\abs{\rho} > 1$, tj. $\mathrm{Re} (\lambda) > 0$, pak $x(t) \xrightarrow{t \rightarrow \infty} \infty$.
		\end{itemize}
	
		\item Stabilita periodického řešení:
		
		mějme dynamický systém daný rovnicí $\dot{x} = f(x)$ s periodickým řešením $\phi(t) = \phi(t+T)$. Uděláme-li linearizaci kolem tohoto řešení, dostaneme úlohu $\dot{\xi} = A(t) \xi$, kde $A(t) = f^{\prime} (\phi(t))$ je $T$ - periodická. Navíc platí $\ddot{\phi} = f^{\prime}(\phi) \dot{\phi} = A(t) \dot{\phi}$, $\dot{\phi}$ je $T$-periodické řešení řešení linearizované úlohy. Tedy pro nelineární $f(x)$ vždy aspoň jeden charakteristický multiplikátor je roven $1$.
		
		\item Ve dvoudimenzionálním ($n = 2$) prostoru lze použít tzv. \textit{Bendixsonovo kritérium}, které poskytuje více informace o periodických řešeních v $\mathbb{R}^{2}$. Toto kritérium bude dále pouze vysloveno a dokázáno později v rámci předmětu \textit{01MMNS}.
	\end{enumerate}
\end{remark}

\medskip

\begin{thrm}\label{thrm:bendixson}
	Nechť $f: \Gamma \rightarrow \mathbb{R}^{2}$, $\Gamma \subset \mathbb{R}^{2}$ je oblast, $f \in \mathrm{C}^{(1)}(\Gamma)$ a $\mathrm{dom}(f) \subset \Gamma$ je jednoduše souvislá množina. Pokud výraz $\mathrm{div}(f)$ pro $x \in \mathrm{dom}(f)$ není identicky roven $0$ a na  $\mathrm{dom}(f)$ nemění znaménko, nemá úloha $\dot{x} = f(x)$ pro počáteční podmínku $x(0) = x_{\mathrm{ini}} \in \mathrm{dom}(f)$ periodické řešení s trajektoriemi ležícími zcela v $\mathrm{dom}(f)$.
\end{thrm}

\medskip

\begin{exmp}\label{exmp:nonlinear_oscillator}
	Nelineární oscilátor je popsán následující rovnici:
	
	\begin{equation}\label{eq:nonlinear_oscillator_1}
		\ddot{\Theta} + p(\Theta) \dot{\Theta} + q(\Theta) = 0,
	\end{equation}
	
	\noindent
	kde $p$ a $q$ jsou hladké, $p(x) > 0$. Pomocí substituce $x_{1} = \Theta$, $x_{2} = \dot{\Theta}$ převedeme \eqref{eq:nonlinear_oscillator_1} na soustavu lineárních ODR:
	
	\begin{equation}\label{eq:nonlinear_oscillator_2}
		\begin{split}
			\dot{x}_{1} &= x_{2}, \\
			\dot{x}_{2} &= -p(x_{1}) x_{2} - q(x_{1}).
		\end{split}
	\end{equation}
	
	Potom $\mathrm{div}(f(x)) = \underbrace{ \frac{ \partial x_{2} }{ \partial x_{1} } }_{ = 0 }  - p(x_{1}) < 0$. Věta \ref{thrm:bendixson} implikuje, že periodická řešení neexistují.
\end{exmp}
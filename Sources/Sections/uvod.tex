\section{Úvod}\label{sec:uvod}

Floquetova teorie zkoumá lineární diferenciální rovnice tvaru  $\dot{\xi } = A(t) \xi$, které se obecně objevují při řešení úloh ve variacích jak pro autonomní, tak i pro neautonomní dynamické systémy. $A$ je v tomto případě periodická s periodou $T$ matice vyjádřující Jacobiho matici pravé strany $f(x)$ dynamického systému $\dot{x} = f(x)$.  Tato teorie poskytuje matematický aparát pro analýzu existence a stability periodických řešení.

\section{Úvodní vztahy a definice}\label{sec:uvodni_vztahy}

\begin{defn} (Fundamentální matice) \\
	Nechť $(y_{1}, y_{2}, \dots y_{n})$ je systém řešení pro rovnici $\dot{\xi } = A(t) \xi$. Pokud $y_{1}, y_{2}, \dots y_{n}$ jsou lineárně nezávislé, pak matice 
	\begin{equation}
		\begin{pmatrix} 
			y_{1}^{1} & y_{2}^{1} & \dots & y_{n}^{1} \\ 
			\vdots & \ddots & & \vdots \\ 
			y_{1}^{n} & y_{2}^{n} & \dots & y_{n}^{n} 
		\end{pmatrix} 
	\end{equation}
	je fundamentální matice dynamického systému.
\end{defn}


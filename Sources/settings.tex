%%%%%% TODO NOTES %%%%%%
\newcommandx{\unsure}[2][1=]{\todo[linecolor=red,backgroundcolor=red!25,bordercolor=red,#1]{#2}}
\newcommandx{\change}[2][1=]{\todo[linecolor=blue,backgroundcolor=blue!25,bordercolor=blue,#1]{#2}}
\newcommandx{\info}[2][1=]{\todo[linecolor=OliveGreen,backgroundcolor=OliveGreen!25,bordercolor=OliveGreen,#1]{#2}}
\newcommandx{\improvement}[2][1=]{\todo[linecolor=Plum,backgroundcolor=Plum!25,bordercolor=Plum,#1]{#2}}
\newcommandx{\thiswillnotshow}[2][1=]{\todo[disable,#1]{#2}}
%% Example:
%%%%%%\todo[inline]{The original todo note withouth changed colours.\newline Here's another line.}
%%%%%%\unsure{Is this correct?}\unsure{I'm unsure about also!}
%%%%%%\change{Change this!}
%%%%%%\info{This can help me in chapter seven!}
%%%%%%\improvement{This really needs to be improved!\newline\newline What was I thinking?!}
%%%%%%\thiswillnotshow{This is hidden since option `disable' is chosen!}
%%%%%%\improvement[inline]{The following section needs to be rewritten!}
%%\newpage
%%\listoftodos[Notes]

\makeatletter
%% Textclass specific LaTeX commands.
\newenvironment{lyxlist}[1]
{\begin{list}{}
{\settowidth{\labelwidth}{#1}
 \setlength{\leftmargin}{\labelwidth}
 \addtolength{\leftmargin}{\labelsep}
 \renewcommand{\makelabel}[1]{##1\hfil}}}
{\end{list}}

%% Larger cdot
\newcommand*{\LargerCdot}{\raisebox{-0.25ex}{\scalebox{1.2}{$\cdot$}}}

%% Algorithms
% Define a \HEADER{Title} ... \ENDHEADER block
\newcommand{\HEADER}[1]{\ALC@it\underline{\textsc{#1}}\begin{ALC@g}}
\newcommand{\ENDHEADER}{\end{ALC@g}}
\renewcommand*{\ALG@name}{Algoritmus}
\algsetup{indent=2em} 
\renewcommand{\algorithmiccomment}[1]{\hspace{2em}// #1} 


%% Footnote style
%\counterwithout{footnote}{section}
\renewcommand{\thefootnote}{\arabic{footnote}}
%\renewcommand{\thefootnote}{\fnsymbol{footnote}}
%\footnote[num]{text}
%instead of num you can put the number of the symbol you like:
%1   asterisk    *   2   dagger  †   3   double dagger   ‡
%4   section symbol  §   5   paragraph   ¶   6   parallel lines  \\
%7   two asterisks   **  8   two daggers ††  9   two double daggers  ‡‡

%% Definitory equality sign
\newcommand\defeq{\mathrel{\stackrel{\makebox[0pt]{\mbox{\normalfont\tiny def.}}}{=}}}

%% Drawing circles
\newcommand{\tikzcircle}[2][black,fill=red]{\tikz[baseline=-0.5ex]\draw[#1,radius=#2] (0,0) circle ;}%
\definecolor{mgcyan}{RGB}{0,255,255}
\definecolor{mgmagenta}{RGB}{255,0,255}

%% Use Times New Roman font for text and Belleek font for math
%% Please make sure that the 'esint' package is turned off in the
%% 'Math options' page.
\usepackage[varg]{txfonts}

%% Use Utopia text with Fourier-GUTenberg math
%\usepackage{fourier}

%% Bitstream Charter text with Math Design math
%\usepackage[charter]{mathdesign}


%% Make the multiline figure/table captions indent so that the second
%% line "hangs" right below the first one.
%\usepackage[format=hang]{caption}

%% Indent even the first paragraph in each section
\usepackage{indentfirst}

%%---------------------------------------------------------------------

%% Disable page numbers in the TOC. LOF, LOT (TOC automatically
%% adds \thispagestyle{section} if not overriden
%\addtocontents{toc}{\protect\thispagestyle{empty}}
%\addtocontents{lof}{\protect\thispagestyle{empty}}
%\addtocontents{lot}{\protect\thispagestyle{empty}}

%% Shifts the top line of the TOC (not the title) 1cm upwards 
%% so that the whole TOC fits on 1 page. Additional page size
%% adjustment is performed at the point where the TOC
%% is inserted.
%\addtocontents{toc}{\protect\vspace{-1cm}}


% completely avoid orphans (first lines of a new paragraph on the bottom of a page)
\clubpenalty=9500

% completely avoid widows (last lines of paragraph on a new page)
\widowpenalty=9500

% disable hyphenation of acronyms
\hyphenation{CDFA HARDI HiPPIES IKEM InterTrack MEGIDDO MIMD MPFA DICOM ASCLEPIOS MedInria}

%%---------------------------------------------------------------------

%% Print out all vectors in bold type instead of printing an arrow above them
\renewcommand{\vec}[1]{\boldsymbol{#1}}

% Replace standard \cite by the parenthetical variant \citep
%\renewcommand{\cite}{\citep}


%% Theorem styles
\theoremstyle{definition}
\newtheorem{defn}{Definice}[section]
\newtheorem{lemma}{Lemma}[section]
\newtheorem{exmp}{Příklad}[section]
\newtheorem{thrm}{Věta}[section]
\newtheorem{remark}{Poznámka}[section]


%% Norm definition
\newcommand{\norm}[1]{\left\lVert#1\right\rVert}

%% Norm definition
\newcommand{\inprod}[1]{\left<#1\right>}

%% Abs definition
\newcommand{\abs}[1]{\left|#1\right|}

%% Space after theorems
\makeatletter
\def\thm@space@setup{%
  \thm@preskip=\parskip \thm@postskip=0pt
}

\makeatother

%% Language and date
%\def\documentdate{August 31, 2018}
\def\documentdate{\today}
\selectlanguage{czech}
